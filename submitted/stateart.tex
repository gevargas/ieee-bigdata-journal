Urban computing is a process of acquisition, integration, and analysis of big and heterogeneous data generated by a diversity of sources in urban spaces, such as sensors, devices, vehicles, buildings, and human, to tackle the major issues that cities face. Urban computing connects unobtrusive and ubiquitous sensing technologies, advanced data management and analytics models, and novel visualization methods, to create win-win-win solutions that improve urban environment, human life quality, and city operation systems.

Urban computing draws is related to the domains of wireless and sensor networks, information science, and human-computer interaction. Collections of devices are used to gather data about the urban environment to help improve the quality of life for people affected by cities. A particular characteristic of urban computing  is the variety of devices, inputs, and human interaction involved. In traditional sensor networks, devices are often purposefully built and specifically deployed for monitoring certain phenomenon such as temperature, noise, and light. In Urban computing in contrast there are different ways of collecting data that comes with different degrees of variety, at different rates, and from devices that are not necessarily aware that they contribute to urban computing applications.

%\subsection{Urban computing frameworks}

\subsection{Urban data}

\subsection{Urban data management techniques}

\subsection{Applications in urban computing}
