Urban computing is a process of acquisition, integration, and analysis of big and heterogeneous data generated by a diversity of sources in urban spaces, such as sensors, devices, vehicles, buildings, and human, to tackle the major issues that cities face. Urban computing connects unobtrusive and ubiquitous sensing technologies, advanced data management and analytics models, and novel visualization methods, to create win-win-win solutions that improve urban environment, human life quality, and city operation systems.

Urban computing draws is related to the domains of wireless and sensor networks, information science, and human-computer interaction. Collections of devices are used to gather data about the urban environment to help improve the quality of life for people affected by cities. A particular characteristic of urban computing  is the variety of devices, inputs, and human interaction involved. In traditional sensor networks, devices are often purposefully built and specifically deployed for monitoring certain phenomenon such as temperature, noise, and light. In Urban computing in contrast there are different ways of collecting data that comes with different degrees of variety, at different rates, and from devices that are not necessarily aware that they contribute to urban computing applications.

%\subsection{Urban computing frameworks}

%............................................................................
\subsection{Urban data}
%............................................................................

Nowadays, sensing technologies and large-scale computing infrastructures have produced a variety of big data in urban spaces, e.g., human mobility, air quality, traffic patterns, and geographical data.

The data generated by traditional sensors is well structured, explicit, clean and easy to understand. However, the data contributed by users is usually in a free format, such as texts and images, or cannot explicitly lead us to the final goal as if using traditional sensors. Sometimes, the information from human sensors is also quite noisy.

%............................................................................
\subsection{Urban data management techniques}
%............................................................................

(Jagadish et al. 2014) proposes a big data infrastructure based on five steps: data acquisition, data cleaning and information extraction, data integration and aggregation, big data analysis and data interpretation. (Chen et al. 2014) use Hadoop-gis to get information on demographic composition and health from spatial data. (Lin and Ryaboy 2013) present their experience on twitter to extract information from log information. They concluded that an efficient big data infrastructure is a balancing speed of development, ease of analysis, flexibility and scalability. Proposing a big data infrastructure on the cloud will make developing big data infrastructures more accessible to small businesses for several reasons: little initial investment, ease of development throw Service-Oriented Architecture (SOA) and using services developed by specialist of each service.

 Advanced management methodology that can well organize multi-modal data (such as streaming, geospatial, and textual data) is still missing. So, computing with multiple heterogeneous data is a fusion of data and also a fusion of algorithms.

%------------------------------------------------------------------------- 
 \paragraph{Spatio-temporal data} 
%-------------------------------------------------------------------------
 Road network data may be the most frequently used geographical data in urban computing scenarios, e.g., traffic monitoring and prediction [Pan and Zheng et al. 2013], urban planning [Zheng et al. 2011b], routing [Yuan and Zheng et al. 2010a; 2011b; 2013b], and energy consumption analysis [Zhang et al. 2013]. Spatio-temporal data is usually represented by a graph that is comprised of a set of edges (denoting road segments) and a collection of nodes (standing for road intersections). Each node has a unique geospatial coordinates; each edge is described by two nodes (sometimes also called terminals) and a sequence of intermediate geospatial points. Other properties, such as a length, speed constraint, type of road, and number of lanes, are associated with an edge.
 
 %------------------------------------------------------------------------- 
 \paragraph{POI} 
%-------------------------------------------------------------------------
 A POI, such as a restaurant and a shopping mall, is usually described by a name, address, category, and a set of geospatial coordinates. While there are massive POIs in a city, the information of POIs could vary in time (e.g., a restaurant may change its name, or be moved to a new location, or be even shut down). POI can be defined out of yellow pages data, they can be manually collected for example using a GPS logger to record the geographic coordinates of a POI by map data providers like Navinfo and AutNavi. They can also be defined by location-based social networking services (e.g., Foursquare). Land use data describes the function of a region, such as residential areas, suburban, and forests, originally planned by urban planners and roughly measured by satellite images in practice. As the satellite image cannot differentiate between fine-grained land use categories, such as educational, commercial, and residential areas, obtaining the current land use data of a big city is not easy [Yuan and Zheng et al. 2012a].
 
%------------------------------------------------------------------------- 
 \paragraph{Traffic and commuting data} 
%-------------------------------------------------------------------------


%------------------------------------------------------------------------- 
 \paragraph{Environmental and natural resources monitoring data} 
%-------------------------------------------------------------------------

%------------------------------------------------------------------------- 
 \paragraph{Personal data} 
%-------------------------------------------------------------------------

%............................................................................
\subsection{Applications in urban computing}
%............................................................................

(Yuan et al. 2013), (Ge et al. 2010), (Lee et al. 2004) worked a transport project to help taxi companies optimize their taxi usage. They work on optimising the odds of a client needing a taxi to meet an empty taxi, optimizing travel time from taxi to clients, based on historical data collected from running taxis. Using knowledge from experienced taxi drivers, they built a mapping of the odds of passenger presence at collection points and direct the taxis based on that map. These research works don’t use real-time data thus making it complicated to make accurate predictions and react to unexpected events. They also use data limited to GPS and taxi usage, whereas other data sources could be accessed and used.

Transdec (Demiryurek et al. 2010) is a project of the University of California to create a big data infrastructure adapted to transport. It’s built on three tiers comparable to the MVC (Model, View, Controller) model for transport data. The presentation tier, based on GoogleTM Map, provides an interface to create the queries and expose the result, the query interface provides standard queries for the presentation tier and a data tier is spatiotemporal database built with sensor data and traffic data. This work provides an interesting query system taking into account the dynamic nature of town data and providing time relevant results in real-time. Urban insight (Artikis et al. 2013) is a European project studying European town planning. In Dublin they are working event detection through big data, in particular on an accident detection system using video stream for CCTV (Closed Circuit Television) and crowdsourcing. Using data analysis they detect anomalies in the traffic and identify if it’s an accident or not. When there is an ambiguity they rely on crowdsourcing to get further information. The RITA (Thompson et al. 2014) project in the United States is trying to identify new sources of data provided by connected infrastructure and connected vehicles. They work to propose more data sources usable for transport analysis. (Jian et al. 2008) propose a service-oriented model to encompass the data heterogeneity of several Chinese towns. Each town maintains its data and a service that allows other towns to understand their data. These services are aggregated to provide a global data sharing service. These papers propose methodologies to acknowledge data veracity and integrate heterogeneous data into one query system. An interesting line to work on would be to produce predictions based on this data to build interesting decision support systems. 

